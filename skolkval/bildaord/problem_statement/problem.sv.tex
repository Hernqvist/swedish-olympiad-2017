\problemname{Bilda ord}

Fatimeh studerar sitt hemspråk som använder det arabiska skriftspråket. Just nu
sitter hon med en övning att bilda ord från givna bokstäver. Hon vill veta på
hur många sätt hon kan bilda ett ord med de givna bokstäverna i uppgiften.

I svenskans skriftspråk skulle övningen kunna se ut så här:

\vspace{0.2in}
\centerline{l s a t}
\vspace{0.2in}

Du får alltså 4 givna bokstäver. Fatimeh vet då att hon måste testa $4*3*2*1 =
24$ permutationer. Skulle istället bokstaven ``s'' vara ett stort ``S'', då vet vi
att vi måste placera bokstaven ``s'' i meningens början. Med det villkoret är
antalet permutationer $3*2*1 = 6$.  Exempelvis kan ordet $Salt$ bildas nu men
inte $laSt$. I det arabiska skriftspråket har man inte stora och små bokstäver
på samma sätt och därför ser restriktionerna annorlunda ut.

\section*{Notation och Indata}

Med bokstäverna i det arabiska skriftspråket har man mycket mer kontext om var
i ordet bokstaven kan förekomma, även i relation till andra bokstäver.  I denna
uppgift har vi $2$ typer av restriktioner, antingen måste en bokstav komma precis före
någon annan bokstav, eller så kan en bokstav bara stå på vissa positioner. Till
exempel

\begin{itemize}
    \item Bokstav $B$ måste stå på på plat $1$ eller plats $4$
    \item Bokstav $D$ måste direkt föregå $C$ eller $B$
\end{itemize}

För att indatat ska vara enkelt att läsa så är notationen formaterad såhär:

\begin{lstlisting}
    B@01,04
    D:CB
\end{lstlisting}

För programkörningen kommer första indataraden bestå av först antalet ord $N$
sedan antalet restriktioner $K$. Därefter följer $K$ rader vardera beskrivande
en restriktion på det sett som förklarats med exempel ovan. Se även
exempelkörningarna.

\section*{Utdata}

Du ska skriva ut ett heltal. Antalet sätt bokstäverna kan
placeras ut. Svaret kommer alltid att understiga 10 miljoner.

\section*{Poängsättning}

För testfall värda upp till $3$ poäng, kommer $N$ vara som mest 9. För full
poäng så ska ditt program klara $N$ som mest 15. $K$ kommer aldrig överstiga $N$.
För stora $N$ kommer restriktionerna var ungefär jämnt utspridda bland
bokstäverna.

\section*{Förklaring exempel 1}

ACDB
ADCB
BADC
CADB
DCAB
BDCA
