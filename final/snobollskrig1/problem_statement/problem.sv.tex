\problemname{Snöbollskrig 1}

I en \href{https://sv.wikipedia.org/wiki/Graf_(grafteori)}{oriktad, viktad graf}
med $N$ noder leker $L$ länder snöbollskrig under IOI.
I början har varje land ett fort i någon nod.

Vid tid $0$ börjar varje land ge sig ut på snöbollskrig. Snöbollskrig fungerar såhär:

\begin{itemize}
\item Om ett land äger ett fort beger sig de tävlande ut längs alla kanter från fortet, med hastighet 1/s.
\item Om två länder möts längs en kant stannar länderna och krigar.
\item Om två länder möts i en nod stannar länderna och krigar.
\item Om ett land når en nod före någon annan bygger det landet ett fort i noden.
\end{itemize}

Avgör vilka par av länder som kommer kriga mot varandra.

\section*{Indata}

Den första raden innehåller tre heltal $N,L,M$ ($2 \le N \le 200\,000, 2 \le L \le 50, 1 \le M \le 500\,000$):
antal noder, antal länder och antal kanter i grafen, respektive.
Därefter följer $L$ rader med heltal, som säger vilken nod varje lands startbas är på.
Sist följer $M$ rader med vardera tre heltal $a, b, w$ ($0 \le a,b < N, 1 \le w \le 1000$).
Detta betyder att det finns en (oriktad) kant mellan noder $a$ och $b$ (noll-indexerade), av längd $w$.

Inga två länder kommer att starta på samma nod.

\section*{Utdata}

För varje par av länder $a, b$ som krigar ska du skriva ut en rad $a b$, där $a < b$ och länderna indexeras från $0$.
Dessa ska skrivas ut i sorterad ordning, sorterat efter den första indexet först.
T.ex., om $L = 3$ och alla tre länder krigar ska du skriva ut
\begin{lstlisting}
0 1
0 2
1 2
\end{lstlisting}

\section*{Förklaring av exempel 1}

I detta exempel är grafen cyklisk och nästan helt symmetrisk. I fallet mellan
land 0 och land 3 kommer kriget utspelas på bågen, medan de 3 andra krigen
kommer utspelas på en nod.

\section*{Poängsättning}

Din lösning kommer att testas på en mängd testfallsgrupper. För att få poäng för en grupp
så måste du klara alla testfall i gruppen.

\begin{tabular}{| l | l | l |}
\hline
% Grupp & Poängvärde & Gränser & Övrigt \\ \hline
Grupp & Poängvärde & Situation \\ \hline
1     & 30         & Grafen är en linje: kanterna är exakt (0,1), (1,2), etc., (N-2, N-1) \\ \hline
2     & 30         & Inga krig förekommer på noderna. \\ \hline
3     & 40         & Inga begränsningar. \\ \hline
\end{tabular}
