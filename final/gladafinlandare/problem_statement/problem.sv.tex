\problemname{Bastubad}
Finländare älskar att bada bastu. Det är inte så konstigt egentligen.
Bastu är både skönt, och bastubadande sker oftast i trevligt sällskap.

I sällskap uppstår dock ofta problem. Det viktigaste när man badar bastu är att temperaturen sätts rätt.
En finländare är väldigt petig med bastuns temperatur.
Olika finländare har olika högstatemperaturer som de tolererar.
I temperaturintervallet som en viss finländare klarar blir finländaren dessutom olika glad beroende på temperaturen.
En finländares glädge ges av en kvadratisk funktion (på formen $ax^2 + bx + c$).

Ett stort sällskap finländare ska nu bada bastu, och behöver din hjälp.
Kan du bestämma den maximala summan av finländarnas glädje, om bastuns temperatur sätts optimalt?
Temperaturen anges i grader kelvin, med en undre gräns på $0 \textrm{ K}$ och övre gräns $100\,000 \textrm{ K}$.

\section*{Indata}
Den första raden innehåller ett heltal $1 \le N \le 100\,000$: antalet finländare.
Därefter följer $N$ rader, en per finländare.
Varje rad innehåller fyra heltal $a, b, c, t$ ($-10^9 \le a,b,c \le 10^9, 1 \le t \le 100\,000$), vilket representerar att finländaren har glädjefunktion $ax^2 + bx + c$, and bara klarar av temperaturer mellan $0\textrm{ K}$ och $t\textrm{ K}$, inklusive.
Funktionen garanteras vara positiv överallt mellan $0$ och $t$.

\section*{Utdata}
Skriv ut ett enda tal: den största möjliga lycka som kan uppnås om temperaturen sätts rätt.
Talet ska skrivas ut med precision minst $10^{-5}$.

\section*{Poängsättning}
\begin{tabular}{| l | l | l |}
\hline
Grupp & Poängvärde & Situation \\ \hline
1     & 21         & $N \le 1000$, alla $t$ är samma. \\ \hline
2     & 29         & $N \le 1000$, alla $a$ är positiva (d.v.s., finnarna gillar temperaturextremer). \\ \hline
3     & 38         & $N \le 1000$, alla $a$ är negativa. \\ \hline
4     & 12         & Inga begränsningar. \\ \hline
\end{tabular}
