\problemname{Kubiska boxar}

Paulo R har insett att alla hans boxar där hemma tar plats. Nu har han bestämt
sig för att stoppa in dem i varandra. Han har $N$ kubiska boxar med
heltalssidlängder $l_1$, $l_2$, $\ldots$, $l_N$. Boxarna har lock så man kan stoppa in
dem i varandra. Dessutom har varje box en färg, R, G eller B. 
En box kan stoppas in i en annan om dess sidlängd är mindre.

Men Paulo inser värdet i att inte bara stoppa in dem i varandra hur som helst,
för då hittar man aldrig dem igen.
Planen är nu att sätta upp en regel som säger vilken färg som kan stoppas i
vilken. Regeln är på formen ''$f_1$ i $f_2$, $f_2$ i $f_3$'' , 
där $f_1$, $f_2$ och $f_3$ är olika färger. Till exempel ''R i
G, G i B''. Notera att man aldrig kommer kunna ha mer än 3 boxar i
varandra. Efter att man bestämt regel så stoppar man boxarna i varandra på det sätt
som gör att det blir så få ytterboxar som möjligt.

Skriv ett program som givet boxarna säger vilken färgregel som ger minst antal
ytterboxar, och sedan skriver ut antalet ytterboxar det kommer bli.

\section*{Indata}

Indat börjar med talet $N$, antalet boxar, där $1\le N \le 1000$. Därav följer $N$ rader med sidlängd
($1 \le l_i \le 10'000'000$) och färg $f_i$ för varje box.

\section*{Utdata}

Du ska alltid skriva ut 3 rader. De första två raderna beskriver färgregeln
och den sista raden antalet ytterboxar det kommer bli efter man stoppat dem in i
varandra. Se exemplen för exakt formatering. Om det finns mer än en färgregel 
som ger minimala antalet ytterboxar kan du skriva ut vilken som helst av dem.

\section*{Poängsättning}

\begin{tabular}{| l | l | l |}
\hline
% Grupp & Poängvärde & Gränser & Övrigt \\ \hline
Grupp & Poängvärde & Situation \\ \hline
1     & 50         & Regeln kommer alltid att vara "R i G, G i B" \\ \hline
2     & 50         & Inga begränsningar. \\ \hline
\end{tabular}
