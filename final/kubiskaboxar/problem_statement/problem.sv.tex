\problemname{Kubiska boxar}

Paulo R har insett att alla hans boxar där hemma tar plats. Nu har han bestämt
sig för att stoppa in dem i varandra. Han har $N$ kubiska boxar med
heltalssidlängder $l_1$, $l_2$ \ellipsis $l_N$. Boxarna har ett tak så man kan stoppa in
dem i varandra. Dessutom har varje box en färg, R, G eller B.

Men Paulo inser värdet i att inte bara stoppa in dem i varandra hur som helst.
För då hittar man aldrig dem igen och det tar lång tid att packa upp etcetera.
Planen är nu att sätta upp en regel som säger vilken färg som kan vara i
vilken. Regeln är på formen ''färg$_1$ i färg$_2$, färg$_2$ i färg$_3$''. Till exempel ''R i
G, G i B''. Det innebär att man aldrig kommer kunna har mer än 3 boxar i
varandra. Efter man bestämt regel så stoppar man boxarna i varandra på det sätt
som möjliggör att det blir så få ytterboxar som möjligt.

Skriv ett program som givet boxarna säger vilken färgregel som ger minst antal
boxar, och sedan skriver ut antalet ytterboxar det kommer bli.

\section*{Indata}

Indat börjar med talet $N < 1000$, antalet boxar. Därav följer $N$ rader med sidlängd
$l_i < 10'000'000$ och färg $f_i$.

\section*{Utdata}

Du ska alltid skriva ut 3 rader. Dem första två raderna beskriver färgregeln
och den sista raden antalet boxar det kommer bli efter man stoppat dem in i
varandra. Se exempelena för exakt formatering.

\section*{Poängsättning}

\begin{tabular}{| l | l | l |}
\hline
% Grupp & Poängvärde & Gränser & Övrigt \\ \hline
Grupp & Poängvärde & Situation \\ \hline
1     & 50         & Regeln kommer alltid att vara "R i G, G i B" \\ \hline
2     & 50         & Inga begränsningar. \\ \hline
\end{tabular}
