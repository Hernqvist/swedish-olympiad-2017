\problemname{Kubiska boxar}
Paulo R har insett att alla hans boxar där hemma tar för mycket plats. Därför har han nu
bestämt sig för att stoppa in dem i varandra. Han har $N$ kubiska boxar med
heltalssidlängder $l_1$, $l_2$, $\ldots$, $l_N$. Boxarna är ihåliga, så man kan stoppa in
dem i varandra. Dessutom har varje box en färg: R, G eller B. 
En box kan stoppas in i en annan om dess sidlängd är mindre.

Men Paulo inser värdet i att inte bara stoppa in boxarna i varandra hur som helst,
för då hittar han dem aldrig igen.
Planen är nu att sätta upp en regel som säger vilken färg som kan stoppas i
vilken. Regeln ska vara på formen ''$f_1$ i $f_2$, $f_2$ i $f_3$'', 
där $f_1$, $f_2$ och $f_3$ är olika färger. Till exempel ''R i
G, G i B''. Notera att han aldrig kommer kunna ha mer än 3 boxar i
varandra. Efter att han bestämt regel så placerar han boxarna i varandra på det sätt
som minimerar antalet ytterboxar (dvs boxar som inte ligger inuti en annan box).

Skriv ett program som givet boxarna säger vilken färgregel som ger minst antal
ytterboxar, och sedan skriver ut antalet ytterboxar det kommer bli.

\section*{Indata}

Indatan börjar med talet $N$, antalet boxar, där $1\le N \le 1\,000$. Därefter följer $N$ rader med sidlängd
$l_i$ ($1 \le l_i \le 10\,000\,000$) och färg $f_i$ (R, G, eller B) för varje box.

\section*{Utdata}

Totalt tre rader ska skrivas ut. Den första raden ska bestå av den första delen
av färgregeln, på formen ''$f_1$ i $f_2$'', medan den andra raden ska bestå av 
den andra delen av regeln, på formen ''$f_2$ i $f_3$''. Notera alltså att den sista bokstaven på
rad $1$ måste vara samma som den första bokstaven på rad $2$.
Den tredje och sista raden ska bestå av ett heltal, som beskriver antalet ytterboxar det kommer bli
efter man stoppat in dem i varandra. Om det finns mer än en färgregel 
som ger minimala antalet ytterboxar kan du skriva ut vilken som helst av dem.

\section*{Poängsättning}
Din lösning kommer att testas på en mängd testfallsgrupper. För att få poäng för en grupp
så måste du klara alla testfall i gruppen.

\noindent
\begin{tabular}{| l | l | l |}
\hline
% Grupp & Poängvärde & Gränser & Övrigt \\ \hline
Grupp & Poängvärde & Gränser \\ \hline
1     & 50         & Regeln kommer alltid att vara "R i G, G i B" \\ \hline
2     & 50         & Inga begränsningar. \\ \hline
\end{tabular}

\section*{Förklaring av exempel 3}
En möjlig lösning är illustrerad i Figur~\ref{fig:sample3}.
Vi kan lägga de två G-boxarna (storlek 1) i varsin R-box (storlek 10).
Därefter kan den ena R-boxen läggas i en B-box.
Kvar återstår då en R-box och en B-box som ytterboxar.
