\problemname{Snöbollskrig 1}

I en viktad graf leker N länder snöbollskrig under IOI. I början har varje land ett fort i någon nod.

Vid tid 0 börjar varje länd ge sig ut på snöbollskrig. Snöbollskrig fungerar som följer: - Om ett land äger ett fort beger sig de tävlande ut längs alla kanter från fortet, med hastighet 1/s
- Om två länder möts längs en kant stannar länderna och krigar
- Om två länder möts i en nod stannar länderna och krigar'
- Om ett land når en nod före någon annan bygger det landet ett fort i noden.

Avgör vilja par av länder som kommer kriga mot varandra.


% Spotify har precis lanserat den nya funktionen Spotify Connect, som möjliggör för
% en användare att fjärrstyra sin uppspelning från mobiltelefonen. Det medför en
% mängd nya tekniska utmaningar, och en av dem är hur loggningen av uppspelningsdata görs.
% För att kunna rapportera till skivbolagen så måste man nämligen veta exakt hur länge
% en användare har lyssnat på musik.

% Du kommer att få loggdata för \texttt{play} och \texttt{paus}-tryckningarna
% för en användare. Användaren använder både sin laptop för att styra musiken, men
% fjärrstyr också ibland med mobiltelefonen via Spotify Connect. Loggarna innehåller både
% datorns och mobilens \texttt{play} och \texttt{paus}-tryckningar. När användaren trycker
% på mobilen så är det exakt $100$ millisekunder delay tills laptopen
% (där musiken spelas upp ifrån) reagerar. Ditt uppdrag är att
% avgöra exakt hur många millisekunder totalt som laptopen spelade upp musik.
% Innan första kommandot utförs så är spelaren i pausat läge.

% \section*{Input}
% Den första raden innehåller heltalet $1 \le N < 1000$, antalet loggrader.

% De efterföljande $N$ raderna innehåller en lista med loggar. Loggarna kommer
% i den ordning knapptryckningar sker med en tidsstämpel i millisekunder, enhet
% (\texttt{laptop} eller \texttt{mobile}) och kommando (\texttt{play} eller \texttt{paus}). Den
% sista loggen kommer
% alltid vara ett \texttt{paus}-kommnado. Dessutom kommer två loggar aldrig ha samma
% tidsstämpel eller ligga exakt 100 millisekunder ifrån varandra.

% För att göra indatat extra lättläst så är loggradernas fält
% alignerade. Tidsstämpeln fylls ut med nollor vänsterifrån så att talet
% alltid blir 7 tecken långt, och mobil stavas istället mobile, alltså med lika
% många tecken som i laptop. Se indataexemplet.

% \section*{Output}
% Ditt program ska skriva ut ett heltal - antalet millisekunder användaren lyssnat på musik.

% \section*{Förklaring av exempel}
% I det första indataexemplet så spelar användaren 500 millisekunder. Sedan använder hen
% mobilen för att starta musiken och avslutar den från laptopen $500$ millisekunder
% senare, men på grund av delayen så spelar musiken bara i 400 millisekunder.
% Totalt spelas alltså musiken i $900$ millisekunder.

% \section*{Poängsättning}
% Din lösning kommer att testas på en mängd testfallsgrupper. För att få poäng för en grupp
% så måste du klara alla testfall i gruppen.

% \begin{tabular}{| l | l | l | l |}
% \hline
% Grupp & Poängvärde & Gränser & Övrigt \\ \hline
% 1     & 50         &  $1 \le N < 1000$ & Alla loggar ligger minst 200 millisekunder ifrån varandra.\\ \hline
% 2     & 50         &  $1 \le N < 1000$ & \\ \hline
% \end{tabular}
